% !TeX root = ../solution.tex

\hypertarget{he22.36}{%
\chapter{[HE22.36] Back to the Roots}\label{he22.36}}

\begin{marginfigure}
	\includegraphics[width=49mm]{level8/challenge36.jpg}
\end{marginfigure}
\section{Intro}
Reversing is considered hard so we thought some old school stuff might be a
gentle start?

This one is only for a simple 8 Bit, 16 MHz and 2KB RAM machine, so how hard
can it be?

Remember: there is always a hard and a simple way.... choose your path!

File: \verb+backtotheroots.zip+

\subsection{Hint}
No hardware is needed for this challenge - though it might be helpful.

\section{Solution}\label{hv22.36solution}

Inspect the binary first:
\begin{itemize}
\item atmega328p
\item openssl aes-128-ecb -d -in secret.txt.enc
\item rockyou.txt
\item Arduino
\end{itemize}

Since I had an Arduino board lying around, load the binary onto it and run it.
It prints a message to the serial port \texttt{Give me your best shot!} and then
sits and waits.  Inspecting the binary in Ghidra a bit more, we see that three
buttons are checked (button2, button4, button3).  Shorting pin 2 triggers a
counter that ends in \texttt{Flag wiped...}.  

Shorting the pins 2, 4, and 3 in short succession prints the flag
\verb+he2022{0ld_Sko0l_CPu$_st1ll_r0cK!}+.

I guess that this was the easy way...
